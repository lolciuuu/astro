% wstep
\setcounter{secnumdepth}{-1}
\renewcommand{\chaptername}{}
\lhead{\emph{Zakończenie}}
\chapter{Zakończenie} 
%============================================================================================================================
%Zakończenie
%============================================================================================================================
\hspace{1cm} W pracy została przedstawiona budowa platformowej gry zręcznościowej mogącej pracować w systemie Linuks, gdzie brakuje gier oraz w najpopularniejszym obecnie systemie dla komputerów osobistych- Windows. 
Gra została stworzona w oparciu o własny silnik, pokazujący możliwości biblioteki SDL. Zaczynając od możliwości stworzenia i zarządzania oknem, poprzez integracje z OpenGL aż do obsługi dźwięku oraz czcionek. Omówiona biblioteka nie ustępuje możliwością bibliotece DirectX, której ogromnym ograniczeniem jest brak wsparcia dla systemu Linuks. Ponadto aplikacja pokazuje przykładowa budowę gry w której do renderowania grafiki wykorzystuje się sprite-y oraz mape kafelkową. Techniki te są jednymi z najbardziej podstawowych zagadnien w grach opartych o grafike dwuwymiarową.

W pracy został poruszony temat rozszerzenia możliwości programu w oparciu o tzw. skryptowanie. Zaprezentowane skrypty w języku Lua stanowią obecnie wtyczki rozszerzające funkcjonalonść nie tylko w grach, lecz wszędzie tam gdzie przeniesienie pewnych informacji poza skompilowany program pozwala oszczędzić czas potrzebny na jego napisanie.  

Stworzona aplikacja nie musi być tylko programem napisanym do pracy dyplomowej, może zostać ona upowszechniona jako projekt Open Source. Nie wykluczone jest, że wtedy będzie dalej rozwijana przez osoby chętne do wolontariatu. Udostępnienie kodu jako darmowego pozwala również na stworzenie pakietu prekompilowanego dla systemu Debian ( Plik z rozszerzeniem deb) i zgłoszenie go do repozytorium dystrybucji Debian, bądź Ubuntu. 

Inną możliwością co do przyszłością programu jest sprzedawanie gry jako tzw. Indie Game czyli gry niezależnej, stworzonej bez wsparcia finansowego przez jedną bądź kilka osób. Taką aplikacje można sprzedawać w formie elektronicznej za niewielkie pieniądze. Aplikacja może zostać uruchomiona na tablecie z systemem BlackBerry i umieszczona w markecie z aplikacjami dla tego systemu. Mowa jest tutaj szczególnie o tabletach ponieważ aplikacja podlega ograniczeniom odnośnie rozdzielczości ekranu na którym może zostać uruchomiona, co zostało opisane bardziej szczegółowo w pracy.  

Efektem ubocznym stworzenia gry Astro Rush było stworzenie silnika gry 2D. Silnik ten dzięki zastosowaniu zewnętrznych skryptów podatny jest na modyfikacje i zmianę. Prosta budowa pozwala na stworzenie na bazie bieżącej aplikacji kolejnej gry. Użycie biblioteki OpenGL do renderowania grafiki daje możliwość stworzenia dużo bardziej zaawansowanej grafiki, wykorzystującej cieniowanie i oświetlenie. 


