% wstep
\setcounter{secnumdepth}{-1}
\renewcommand{\chaptername}{}
\lhead{\emph{Wstęp}}
\chapter{Wstęp} 
%----------------------------------------------------------------------------------------
% Wstep
%----------------------------------------------------------------------------------------
\hspace{1cm} Celem tej pracy jest stworzenie gry działającej w systemie Linuks, na który to obecnie jest nieporównywalnie mniej gier niż dla komercyjnego systemu Microsoft Windows. Linuks obecnie najczęściej znajduje zastosowanie jako oprogramowanie serwera, superkomputera bądź jako system wbudowany. Rośnie jednak jego pozycja jako system dla komputerów biurkowych, chodź tutaj nadal uważany jest za system wyłącznie dla tzw. Geeków, czyli maniaków komputerowych z bardzo dużą wiedzą. Pogląd ten stopniowo zmieniany jest przez takie dystrybucje jak Ubuntu. Jest ona równie łatwa w użytkowaniu dla przeciętnego człowieka jak system Windows. Ciągle jednak Linuks nie cieszy się popularnością wśród graczy i twórców gier. Tendencja ta zaczęła się zmieniać w ciągu kilku ostatnich lat, czego dowodem może być wydanie na Linuksa przez firmę Valve Corporation systemu dystrybucji gier ,,Steam``. Jest to niewątpliwie krok do przodu jeżeli chodzi o zmianę poglądu twórców gier, że na Linuksa nie warto wydawać gier. Warto tutaj wspomnieć jeszcze o tym, że Valve nie jest mało znaną firmą, bardzo wielu graczy kojarzy ją z grą Counter Strike, w którą przez kilka lat grały setki tysięcy ludzi na całym świecie.

Założenie, że aplikacja ma działać natywnie w Linuksie wykluczało użycie narzędzi nie kompatybilnych z tymże systemem, przykładem może być tutaj często wykorzystywany przy tworzeniu gier DirectX firmy Microsoft. Wybór padł natomiast na wieloplatformową bibliotekę Simple Direct Media Layer (w skrócie SDL) której lista docelowych platform jest bardzo długa.
Zaczynając od Linuksa, poprzez Windows, Mac OS aż do takich egzotycznych systemów jak Amiga OS. Dodatkową zaletą biblioteki SDL jest fakt, że stanowi ona wolne oprogramowanie open source na licencji „zlib”. SDL posiada także kilka dodatkowych bibliotek stanowiących rozszerzenie jej możliwości. W aplikacji zostaną wykorzystane dodatkowe moduły rozszerzające API SDL-a o obsługę
dźwięku oraz czcionek. Nie są to jednak wszystkie dostępne wtyczki do SDL-a, warto wspomnieć też o bibliotece SDL\_Net dzięki której możliwe jest stworzenie wieloosobowej  gry sieciowej. Najważniejszymi możliwościami jakimi dysponuje SDL jest utworzenie kontekstu graficznego i obsługa zdarzeń, biblioteka pozwala również za pomocą zestawu funkcji renderować obraz. Rysowanie za pomocą SDL-a często okazuje się jednak zbyt wolne, twórcy biblioteki pozwolili obejść ten problem poprzez wykorzystanie do renderowania 
niskopoziomowej biblioteki OpenGL, która również jest kompatybilna z Linuksem.

Następnym założonym celem aplikacji było wykorzystywanie zewnętrznych skryptów (tzw. skryptowanie) które dawałyby możliwość manipulowania pewnymi danymi aplikacji bez potrzeby jej re-kompilacji. Skrypty te będą napisane w języku Lua, który został zaimplementowany w ANSI C, dzięki czemu zapewnia wysoką wydajność i przenośność na wiele platform. Ogromnym plusem połączenia programu napisanego w C++ oraz Lua jest to, że z poziomu aplikacji C++ można wywoływać funkcje zadeklarowane w skrypcie, a mogą one być zmieniane bez potrzeby rekompilacji całej aplikacji. Funkcje umieszczone w skrypcie są uruchamianie podczas działania aplikacji przez maszynę wirtualną Lua. Cały ten mechanizm działa również w drugą stronę z poziomu skryptu Lua można wywołać funkcję C++, co też zostanie wykorzystane w aplikacji.

Stworzona na potrzeby pracy gra będzie posiadać grafikę 2D, a dedykowanym systemem operacyjnym będzie Linuks, chodź dzięki zastosowaniu wieloplatformowych narzędzi pozostaje możliwość uruchomienia jej w innych systemach np. Microsoft Windows, bądź też Mac OS. Warto tutaj wspomnieć także o mobilnym systemie Blackberry, który to wspiera wszystkie technologie które będą użyte w pracy. Daje to możliwość umieszczenia gry w BlackBerry World – markecie z aplikacjami na urządzenia mobilne z tymże systemem. Co wiąże się z możliwością zarobienia pieniędzy na tej grze. Reasumując, w pracy zostanie przedstawiona aplikacja pokazująca możliwości SDL-a jako biblioteki do tworzenia gier. Przedstawiona zostanie również możliwość wykorzystania skryptów w języku Lua jako narzędzia pozwalającego przenieść część logiki aplikacji poza skompilowany
program.
%----------------------------------------------------------------------------------------

