%----------------------------------------------------------------------------------------
%							ustawienia dokumentu
%----------------------------------------------------------------------------------------

\documentclass[12pt, a4paper, oneside]{Thesis} % rozmiar czcionki,papieru, czy jednostronnie
\graphicspath{{./Pictures/}} % katalog z obrazkami

\usepackage[square, numbers, comma, sort&compress]{natbib} 
\hypersetup{urlcolor=blue, colorlinks=true} % kolory hiperlinków
\title{\ttitle} % Defines the thesis title - don't touch this

% włączenie polskich znaków
\usepackage[T1]{fontenc}
\usepackage[polish]{babel}
\usepackage[utf8]{inputenc}

%dodatkowe pakiety
\usepackage{lmodern}
\usepackage{graphicx}
\usepackage{wrapfig}
\usepackage{hyperref}
\usepackage[table]{xcolor}

%definicja koloru czcionki
\definecolor{TEXT_COLOR}{HTML}{3B3B3B}
%==============================================================================================================
\begin{document}

\color{TEXT_COLOR}
\setstretch{1.5} % odstęp pionowy 

% Define the page headers using the FancyHdr package and set up for one-sided printing
\fancyhead{} % Clears all page headers and footers
\rhead{\thepage} % Sets the right side header to show the page number
\lhead{} % Clears the left side page header

\pagestyle{fancy} % Finally, use the "fancy" page style to implement the FancyHdr headers

\newcommand{\HRule}{\rule{\linewidth}{0.5mm}} % New command to make the lines in the title page

% PDF meta-data
\hypersetup{pdftitle=Kamil Lolo- praca inżynierska}
\hypersetup{pdfsubject=Praca inżynierska}
\hypersetup{pdfauthor=Kamil Lolo}
\hypersetup{pdfkeywords=Praca inżynierska}

\setcounter{secnumdepth}{3}

%----------------------------------------------------------------------------------------
%								Strona tytułowa
%----------------------------------------------------------------------------------------

\begin{titlepage}
\begin{center}

%logo uniwersytetu
\begin{figure}
    \centering
    \includegraphics[width=300px]{./Pictures/logo.jpg}
\end{figure}

\begin{center}
\Large Inżynierska praca dyplomowa \\*
\end{center}

\addvspace{40pt}

{\huge \bfseries Platformowa gra zręcznościowa \\* z wykorzystaniem biblioteki SDL }\\[0.4cm] % tytuł pracy

\addvspace{110pt}
 
\begin{minipage}{0.4\textwidth}
\begin{flushleft} \large
\emph{Autor:}\\
Kamil Lolo 
\end{flushleft}
\end{minipage}
\begin{minipage}{0.4\textwidth}
\begin{flushright} \large
\emph{Promotor:} \\
dr.Krzysztof Podlaski
\end{flushright}
\end{minipage}\\[3cm]


\vfill
\addvspace{50pt}
\begin{center}
Wydział Fizyki i informatyki stosowanej
\end{center}
{\large \today}\\[4cm] % Date


\end{center}
\end{titlepage}

%----------------------------------------------------------------------------------------
%								Spis treści
%----------------------------------------------------------------------------------------

% The page style headers have been "empty" all this time, now use the "fancy" headers as defined before to bring them back
\pagestyle{fancy}

% Set the left side page header to "Contents"
\lhead{\emph{Contents}} 

% Write out the Table of Contents
\tableofcontents 


%----------------------------------------------------------------------------------------
%									Rozdziały pracy
%----------------------------------------------------------------------------------------
% Rozpoczecie numerowania stron (1,2,3..)
\mainmatter 
% Return the page headers back to the "fancy" style
\pagestyle{fancy} 
% dołączanie poszczegolnych rozdzaiałów
% wstep
\setcounter{secnumdepth}{-1}
\renewcommand{\chaptername}{}
\lhead{\emph{Wstęp}}
\chapter{Wstęp} 
%----------------------------------------------------------------------------------------
% Wstep
%----------------------------------------------------------------------------------------
\hspace{1cm} Celem tej pracy jest stworzenie gry która będzie działać w systemie Linuks, na który to obecnie jest nieporównywalnie mniej gier niż dla komercyjnego systemu Microsoft Windows. Linuks obecnie najczęściej znajduje zastosowanie jako oprogramowanie serwera, superkomputera bądź jako system wbudowany. Rośnie jednak jego pozycja jako system dla komputerów biurkowych, chodź tutaj nadal uważany jest za system wyłącznie dla tzw. Geeków, czyli maniaków komputerowych z bardzo dużą wiedzą. Pogląd ten stopniowo zmieniany jest przez takie dystrybucje jak Ubuntu. Jest ona równie łatwa w użytkowaniu dla przeciętnego człowieka jak system Windows. Ciągle jednak Linuks nie cieszy się popularnością wśród graczy i twórców gier. Tendencja ta zaczęła się zmieniać w ciągu kilku ostatnich lat, czego dowodem może być wydanie na Linuksa przez firmę Valve Corporation systemu dystrybucji gier ,,Steam``. Jest to niewątpliwie krok do przodu jeżeli chodzi o zmianę poglądu twórców gier że na Linuksa nie warto wydawać gier. Warto tutaj wspomnieć jeszcze o tym że Valve nie jest mało znaną firmą, bardzo wielu graczy kojarzy ją z grą Counter Strike, w którą przez kilka lat grały setki tysięcy ludzi na całym świecie.

Założenie że aplikacja ma działać natywnie w Linuksie wykluczało użycie narzędzi nie kompatybilnych z tymże systemem, przykładem może być tutaj często wykorzystywany przy tworzeniu gier DirectX firmy Microsoft. Wybór padł natomiast na wieloplatformową bibliotekę Simple Direct Media Layer (w skrócie SDL) której lista docelowych platform jest bardzo długa.
Zaczynając od Linuksa, poprzez Windows, Mac OS aż do takich egzotycznych systemów jak Amiga OS. Dodatkową zaletą biblioteki SDL jest fakt że stanowi ona wolne oprogramowanie open source na licencji „zlib”. SDL posiada także kilka dodatkowych bibliotek stanowiących rozszerzenie jej możliwości. W aplikacji zostaną wykorzystane dodatkowe moduły rozszerzające API SDL-a o obsługę
dźwięku oraz czcionek. Nie są to jednak wszystkie dostępne wtyczki do SDL-a, warto wspomnieć też o bibliotece SDL\_Net dzięki której możliwe jest stworzenie wieloosobowej  gry sieciowej. Najważniejszymi możliwościami jakimi dysponuje SDL jest utworzenie kontekstu graficznego i obsługa zdarzeń, biblioteka pozwala również za pomocą zestawu funkcji renderować obraz. Rysowanie za pomocą SDL-a często okazuje się jednak zbyt wolne, twórcy biblioteki pozwolili obejść ten problem poprzez wykorzystanie do renderowania 
niskopoziomowej biblioteki OpenGL, która również jest kompatybilna z Linuksem.

Następnym założonym celem aplikacji było wykorzystywanie zewnętrznych skryptów (tzw. skryptowanie) które dawałyby możliwość manipulowania pewnymi danymi aplikacji bez potrzeby jej re-kompilacji. Skrypty te będą napisane w języku Lua, który został zaimplementowany w ANSI C, dzięki czemu zapewnia wysoką wydajność i przenośność na wiele platform. Ogromnym plusem połączenia programu napisanego w C++ oraz Lua jest to że z poziomu aplikacji C++ można wywoływać funkcje zadeklarowane w skrypcie, a mogą one być zmieniane bez potrzeby rekompilacji całej aplikacji. Funkcje umieszczone w skrypcie są uruchamianie podczas działania aplikacji przez maszynę wirtualną Lua. Cały ten mechanizm działa również w drugą stronę z poziomu skryptu Lua można wywołać funkcję C++, co też zostanie wykorzystane w aplikacji.

Stworzona na potrzeby pracy gra będzie posiadać grafikę 2D, a dedykowanym systemem operacyjnym będzie Linuks, chodź dzięki zastosowaniu wieloplatformowych narzędzi pozostaje możliwość uruchomienia jej w innych systemach np. Microsoft Windows, bądź też Mac OS. Warto tutaj wspomnieć także o mobilnym systemie Blackberry  który to wspiera wszystkie technologie które będą użyte w pracy. Daje to możliwość umieszczenia gry w Black Berry App Word – markecie z aplikacjami na urządzenia mobilne z tymże systemem. Co wiąże się z możliwością zarobienia pieniędzy na tej grze. Reasumując, w pracy zostanie przedstawiona aplikacja pokazująca możliwości SDL-a jako biblioteki do tworzenia gier. Przedstawiona zostanie również możliwość wykorzystania skryptów w języku Lua jako narzędzia pozwalającego przenieść część logiki aplikacji poza skompilowany
program.
%----------------------------------------------------------------------------------------


\setcounter{secnumdepth}{3}
\renewcommand{\chaptername}{Rozdział}
\chapter{Wprowadzenie} 

%============================================================================================================================
% Czy jest gra komputerowa?
%============================================================================================================================
\section{Czym jest gra komputerowa?}
\lhead{Rozdział 1. \emph{Czym jest gra komputerowa?}}
\hspace{1cm} Gra komputerowa jest specyficznym rodzajem programu komputerowego łączącego w sobie elementy grafiki i dźwięku. Program ten ma za zadanie dostarczać użytkownikowi, bądź też kilku użytkownikom interaktywnej rozrywki przy komputerze, bądź też innym urządzeniem multimedialnym np.konsoli, tablecie. Gry można sklasyfikować na bardzo wiele gatunków i podgatunków np. gry zręcznościowe, gry z grafiką 3D, strategiczne, FPS-y (ang. first person shooter) itd. Można je podzielić również na to do kogo są kierowane tzn. czy gra jest edukacyjną aplikacją dla dzieci, czy też jest brutalną grą wojenną dla dorosłych.

W czasach kiedy powstawały pierwsze gry (lata 70), oraz w czasach kiedy komputery klasy PC dopiero stawały się popularne, gry były tworzone przez jedną bądź kilka osób. Dziś w produkcji gier bierze udział niejednokrotnie kilkanaście osób, a w tym nie tylko programiści ale także projektanci, reżyserowie, dźwiękowcy, graficy. 
Proces ten nabiera często ogromnego rozmachu i przynosi duże pieniądze ze sprzedaży gotowego produktu. Produkcja gry komputerowej jest tak bardzo złożonym procesem że powstają filmy dokumentalne o tym jak się odbywa ten proces. Przykładem może być tutaj film z serii \href{http://natgeotv.com/uk/megafactories/videos/ea-sports}{megafabryki} pokazujący proces Tworzenia gry Fifa- realistycznego symulatora piłki nożnej. 

W dzisiejszych czasach coraz częściej można spotkać się z organizowaniem turniejów w różnych grach komputerowych, a nawet mistrzostwa świata. Współzawodnictwo takie nazywane jest e-sportem, i dla wielu graczy stanowi źródło dochodów. Przykładem może gra League of Legends, w którą grają  \href{http://gamezilla.komputerswiat.pl/newsy/2012/41/league-of-legends-w-liczbach-przerazajaco-wysokich-liczbach}{miliony ludzi na świecie} i regularnie odbywają się zawody w tą grę, oraz istnieją ligi w ramach których gracze zmagają się między sobą.

Gry komputerowe w dzisiejszych czasach to nie tylko rozrywka, ale także duże pieniądze. Przykładem może być tutaj gra Call of Duty Modern Warfare 2 która wygenerowała \href{http://www.cdaction.pl/news-10119/wiedziales-call-of-duty-wygenerowalo-3-miliardy--dolarow-przychodu.html}{setki milionów dochodu}.
 
\begin{figure}[h]
    \centering
    \includegraphics[height=240px]{./Pictures/cryengine.jpg}
    \caption{Silnik Cry Engine}
\end{figure}
  
Omawiając gry komputerowe nie można pominąć zagadnienia silników gier. Silnik gry jest oprogramowaniem które dostarcza twórcy gry gotowych rozwiązań w kwestii np. wykrywania kolizji, oświetlenia, obsługi dźwięku oraz wielu innych elementów. Dzięki zastosowaniu gotowego silnika developer nie traci czasu na pisanie np. modułu do wyświetlania animacji, tylko korzysta z gotowego rozwiązania skupiając się na fabule, bądź też jakości wyświetlanej grafiki. Takie podejście znacząco przyśpiesza Tworzenie gry, oraz zwiększa jej jakość. Osobny zespół pracuje nad silnikiem, a inny nad stworzeniem na jakiego podstawie gry. Duże studia tworzą własne silniki, na bazie którego Tworzą kolejne gry, przykładem może być tutaj Cry Engine dający fotorealistyczną grafikę. Został on wykorzystany do stworzenia polskiej produkcji "Sniper Ghost Warrior 2" korzystająca z  3 wersji tego silnika. Silnik ten dostępny jest do kupienia bądź też do ściągnięcia za darmo ze strony twórców. Silnik posiada środowisko graficzne w którym w sposób wizualny można tworzyć świat dostępny w grze. Co ciekawe taki edytor znajduje zastosowanie nie tylko jako narzędzie do tworzenia gier ale także jako aplikacja do tworzenia realistycznych wizualizacji budynków przez architektów. Minusem większości komercyjnych silników jest jednak brak wsparcia dla systemu Linuks. Ewentualne dostępne dla tego systemu silniki nie dorównują możliwością tym płatnym. Stało się to też przyczyną wybrania grafiki 2D jako podstawy do napisania omawianego projektu, oraz do stworzenia gry od podstaw, bez użycia gotowych rozwiązań. 

%============================================================================================================================
% Fabuła
%============================================================================================================================
\section{Fabuła}
\lhead{Rozdział 1. \emph{Fabuła}}

\hspace{1cm} Omawiany projekt będzie grą zręcznościową, celem gracza będzie przebiec bohaterem jak największy dystans. Bohaterem tym będzie astronauta który wylądował na obcej planecie, i musi on zbierać bańki z tlenem żeby się nie udusić. Bańki te uzupełniają poziom życia wskazywany przez pasek życia w lewym górnym rogu. Dodatkowym utrudnieniem będą przeszkody w postaci meteorytów które należy omijać żeby nie pomniejszyć ilości życia.

 
\begin{wrapfigure}{left}{0.5\textwidth}
\begin{center}
\includegraphics[width=120px]{./Pictures/astro.jpg}
\end{center}
\caption{Postać astronauty }
\label{Etykieta}
\end{wrapfigure}


Astronauta podczas gry będzie cały czas biec do przodu,zwalniając nieznacznie w przypadku niskiego poziomu życia. Zwolnienie te będzie odbywać się poprzez zmianę położenia bohatera względem lewego brzegu ekranu. Prędkość przesuwania mapy nie ulega w tym przypadku zmianom. Poziom życia będzie ciągle spadał w regularnych odstępach czasu, natomiast bańki z tlenem będą go zwiększać. Ilość życia które doda jedna zebrana bańka jest większe niż ilość życia zmniejszona w jednym przy upływie jednego odstępu czasu. Gracz będzie miał możliwość podskakiwania astronautą oraz wznoszenia się nim do góry Sporadycznie na mapie będą się pokazywać bonusy które astronauta będzie mógł zebrać (maksymalnie 3 naraz) i wykorzystać później do uzupełnienia ilości życia. Taki bonus 
będzie dawał także nieśmiertelność przez kilka sekund, wtedy to na brzegach ekranu pojawi się charakterystyczna niebieska obwódka. Kiedy gracz zakończy grę, wtedy jego wynik, czyli ilość przebytych metrów zapisywany będzie na liście 10 najlepszych wyników, o ile ilość punktów będzie większa od najniższego wyniku na liście. Kiedy zostanie spełniony ten warunek, gracz zostanie poproszony o podanie swojego imienia. Wyniki te będą zapisywane w osobnym pliku na dysku, tak żeby dane nie zostały stracone po wyłączeniu aplikacji. Listę najlepszych wyników będzie można obejrzeć wybierając z głównego menu pozycje highscore. 

\begin{figure}[h]
    \centering
    \includegraphics[height=42px]{./Pictures/livebar.png}
    \caption{Pasek życia wyświetlany w lewym górnym rogu}
\end{figure}

Ze względu na fabułę z biegnącym astronautą, oraz osadzenie zdarzeń na obcej planecie, gra została nazwa ,,Astro Rush`` Tytuł ten nawiązuje również do gry Dino Rush dostępnej dla urządzeń z systemem iOS, która byłą inspiracją do stworzenia Astro Rush.

%============================================================================================================================
% Grafika
%============================================================================================================================
\section{Grafika}
\lhead{Rozdział 1. \emph{Grafika}}

\hspace{1cm} Grafika w grze będzie się opierać o technikę tworzenia animacji, gdzie pojedynczy animowany element nazywany jest sprite. Technika ta polega na rysowaniu animacji za pomocą serii klatek które są przechowywane bezpośrednio obok siebie w jednym pliku graficznym. Korzystając z tej techniki można renderować duże obrazki za pomocą serii mniejszych grafik, co pozwala na oszczędzenie miejsca  pamięci, na dysku oraz na zwiększenie wydajności aplikacji. W przypadku gry Astro Rush animowany jest bieg astronauty, poszczególne klatki tej animacji narysowanej w programie Inkscape przedstawia poniższy rysunek.

\begin{figure}[h]
    \centering
    \includegraphics[width=0.8\textwidth,natwidth=410,natheight=142]{./Pictures/astroRun.jpg}
    \caption{Animacja biegu astronauty}
\end{figure}

Wszystkie animacje w grze są przechowywane w jednym głównym pliku „atlas.png”. Skąd wyświetlany jest tylko fragment odpowiadający danemu sprite-owi.
Po upływie określonego czasu następuje przejście do następnej klatki animacji, czyli zazwyczaj przesuniecie współrzędnej X o szerokość obrazka. W tym
algorytmie współrzędna Y nie zmienia się. Cały algorytm wyświetlania animacji opartej przestawia schemat 1.
Warto tutaj wspomnieć o układzie współrzędnych jaki jest używany w bibliotece SDL. Otóż punkt początkowy (0,0) znajduje się w lewym górnym
rogu, prawy górny wierzchołek to ( szerokość okna, 0 ), natomiast lewy dolny to : (0, wysokość okna ). Grafika na potrzeby gry została częściowo
stworzona w edytorze grafiki wektorowej Inkscape, który oparty jest na licencji GPL i działa pod takimi systemami operacyjnymi jak np. Windows, Linux.
Narysowanie części obrazków jako grafiki wektorowej pozwoliło zachować pełną skalowalność w dalszym procesie tworzenia grafiki. Utworzone grafiki
wektorowe były składane i poprawiane w Adobe Photoshop – bardzo rozbudowanej aplikacji do obróbki grafiki rastrowej. Photoshop jest aplikacją płatną,
jednak istnieje możliwość użycia 30 dniowej wersji Trial, co też zostało zrobione podczas tworzenia gry. W atlasie grafiki znalazły się także ikony z
kolekcji „Hand drawn icon set” które autor opublikował w internecie  na darmowej licencji.
\begin{figure}[h]
    \centering
    \includegraphics[width=0.8\textwidth,natwidth=510,natheight=142]{./Pictures/screen1.png}
    \caption{Przepływ danych między aplikacją C++ a skryptem Lua}
\end{figure}

\begin{figure}[h]
    \centering
    \includegraphics[width=0.8\textwidth,natwidth=410,natheight=142]{./Pictures/screen1.png}
    \caption{Przepływ danych między aplikacją C++ a skryptem Lua}
\end{figure}

\begin{figure}[h]
    \centering
    \includegraphics[width=0.8\textwidth,natwidth=510,natheight=142]{./Pictures/sprite_algorytm.png}
    \caption{Screen z gry - astronauta podczas lotu}
\end{figure}

%============================================================================================================================
% g++
%============================================================================================================================
\section{Użyte narzędzia}
\lhead{Rozdział 1. \emph{Użyte narzędzia}}
\subsection{Kompilator}
\hspace{1cm} Do zbudowania aplikacji został wykorzystany kompilator g++ dostępny w ramach GNU GCC - zestawu kompilatorów dla popularnych języków. GCC dostępny na licencji GPL pozwala na kompilacje aplikacji napisanej w min. w Java, C/C++, Ada. Dostępny jest na wiele platform i architektur. Ponadto w systemach Linuks jest często zainstalowany domyślnie np. w Gentoo gdzie każda aplikacja jest kompilowana ze źródeł. G++ posiada kompatybilność z językiem C, dzięki czemu nie ma problemu z kompilacją programu korzystającego z biblioteki SDL, która jest napisana właśnie w C. Kompilator ten jest aplikacją konsolową, a skompilowanie zwykłego pliku źródłowego (main.cpp) do pliku wynikowego (main.bin) można wykonać za pomocą polecenia:
\begin{verbatim}
	g++ -c ./main.cpp -o main.bin
\end{verbatim}

W przypadku aplikacji składającej się z jednego pliku i nie korzystającego z żadnych bibliotek taka kompilacja wydaję się łatwa.
Kiedy natomiast mamy kilka plików do skompilowania, wtedy należy wszystkie je wpisać jako parametr, a dodatkowo trzeba ustawić 
flagi linkera, który łączy skompilowane pośrednie pliki w plik wykonywalny.
Projekt Astro Rush składa się z około 20 klas, wpisywanie takiego polecenia za każdym razem byłoby bardzo nie wygodne, dlatego też zostało użyte narzędzie make usprawniające budowanie gry. Zostanie ono opisane w jednym z kolejnych rozdziałów. 

Warto wspomnieć że g++ posiada bardzo wiele przydatnych parametrów, przy kompilacji projektu zostały wykorzystane opcje: \\*
-\textbf{Wall} - włącza wszystkie ostrzeżenia przy budowaniu. \\*
-\textbf{pedantic} włącza wszystkie ostrzeżenia związane ze standardem ISO języka C/C++ \\*
-\textbf{g} kompilacja z włączonym debuggowaniem. Do skompilowanej aplikacji dołączane są dane potrzebne do debuggowania poprzez gdb. \\*
-\textbf{std=c++0x} flaga wymusza użycie najnowszego standardu języka C++. \\*
-\textbf{O0} Określa poziom optymalizacji. Poziom 0 ustawiony domyślnie wyłącza optymalizacje, przyśpieszając tym samym czas kompilacji. Przed udostępnieniem aplikacji użytkownikowi należy włączyć optymalizacje np. za pomocą -O3, gdzie poziom 3 jest najwyższym poziomem optymalizacji.  

Dodatkowo potrzebne okazały się następujące flagi linkera:
\-GL \-lGLU \-lSDL 


%============================================================================================================================
% Eclipse
%============================================================================================================================

\lhead{Rozdział 1. \emph{Użyte narzędzia}}
\subsection{Dlaczego aplikacja powstawała w Eclipse?}
\hspace{1cm} Środowiskiem w którym będzie powstawać aplikacja będzie Eclipse IDE (ang. Integrated Development Environment). Aplikacja została stworzona przez firmę IBM, jednak obecnie stanowi ona darmowe oprogramowanie, dostępne na licencji Eclipse Public License (EPL). Licencja ta bardzo dobrze nadaje się do wykorzystania w celach komercyjnych, dzięki czemu Eclipse jest często platforma wykorzystywaną do tworzenia oprogramowania w korporacjach. Warto wspomnieć o tym że funkcjonalność aplikacji można rozszerzać za pomocą wtyczek. Eclipse posiada nawet market ( http://marketplace.eclipse.org/ ) w którym niezależni Twórcy mogą sprzedawać stworzone przez siebie plugin-y.
Dostępne są tam zarówno płatne rozszerzenia, jak i darmowe, a samą instalacje można przeprowadzić z poziomu uruchomionej aplikacji wybierając w górnym menu Help a następnie Eclipse Marketplace.


\begin{figure}[h]
    \centering
    \includegraphics[width=0.8\textwidth,natwidth=480,natheight=142]{./Pictures/eclipse.png}
    \caption{Eclipse podczas instalacji wtyczki z Marketplace}
\end{figure}

Domyślnie w wersji klasycznej Eclipse możliwe jest jedynie tworzenie aplikacji w Javie, jednak instalacja wtyczki CDT, bądź też pobranie przygotowanej już odpowiedniej wersji programu pozwala na wygodne pisanie w języku C++.
Konfiguracja tego środowiska do tworzenia gry Astro Rush polegała na pobraniu z oficjalnej strony spakowanej platformy dostosowanej do języka C++, wypakowaniu jej (program wymaga do pracy zainstalowanego JRE, ponieważ napisana jest w Javie), a następnie po uruchomieniu doinstalowaniu kilku dodatkowych wtyczek które okazały się przydatne podczas tworzenia projektu np. wtyczki zapewniającej wsparcie dla repozytorium Git-a. Środowisko te posiada wbudowaną integracje z wieloma przydatnymi narzędziami, które bardzo upraszczają życie programiście np. auto uzupełnienie, formatowanie kodu. Natomiast wtyczka CDT dostarczyła przy tworzeniu gry Astro Rush taki przydatnych narzędzi jak:
-Integracja Eclipse z GNU Debuggerem (w skrócie gdb). Debugger ten stworzony przez Richarda Stallmana i dostępny na licencji GPL jest aplikacją konsolową która umożliwia debuggowanie aplikacji C++ w Eclipse. Najbardziej przydatna okazała się tutaj możliwość zatrzymania programu na ustawionym breakpoincie oraz sprawdzenie stosu wywołań. Można również wykonywać kod programu krok po kroku, szukając przyczyny ewentualnych błędów. Używanie gdb z poziomu konsoli byłoby bardziej uciążliwe. 
-Integracja z Make (samo narzędzie zostanie omówione w kolejnym podrozdziale) dzięki czemu nie ma potrzeby wpisywania polecenia make w konsoli za każdym razem przy budowaniu, ale wystarczy w oknie Eclipsa wybrać cel budowania i dwa razy kliknąć w niego.

Trzeba również wspomnieć o tym że przy tworzeniu projektu został wykorzystany plugin integrujący Eclipse z programem Valgrind. 
Ta aplikacja konsolowa jest narzędziem do debugowania pamięci oraz do wykrywania wycieków pamięci. Plugin ten okazał się pomocny wielokrotnie do wyszukania miejsc w aplikacji gdzie dynamicznie przydzielana pamięć nie była zwalniania. Valgrind pozwala nawet znaleźć zmienne które nie są inicjalizowane, co pozwala uniknąć niektórych błędów związanych z tym że zmienne automatyczne zawierają po utworzeniu przypadkowe wartości z pamięci. 

W ostatniej fazie tworzenia projektu zostało wykonane profilowanie, czyli dynamiczna analiza aplikacji pozwalająca znaleźć miejsca aplikacji których wywołanie trwa najdłużej w celu ich optymalizacji. Do tego celu został wykorzystany plugin łączący Eclipse z profilerem Perf. Pozwolił on na zebranie i wyświetlenie statystyk odnośnie czasu procesora jaki jest wykorzystywany przez poszczególne funkcje, klasy w programie.

%============================================================================================================================
% Make jako narzędzie do automatyzacji budowania projektu
%============================================================================================================================
\subsection{Make jako narzędzie do automatyzacji budowania projektu}
\hspace{1cm} Eclipse jako platforma programistyczna dedykowany jest językowi Java, a kolejne wtyczki pozwalające pisać w innych językach to tylko rozszerzenie tego środowiska Javy. Eclipse z wtyczką do języka C++ może używać programu Make, który automatyzuje budowanie projektu składającego się z wielu plików. Zostało to wykorzystane w projekcie. Ogromnym plusem takiego rozwiązania jest to żeby zbudować aplikacje u klienta nie potrzeba ściągać Eclipse-a i konfigurować projektu, ale wystarczy aplikacja make, która zajmuje niewiele miejsca na dysku i  wystarczy wykonać jedno polecenie żeby zbudować projekt. Rozwiązanie takie jest nie zależne od systemu, ponieważ make działa zarówno w systemie Windows jak i Linux. Dodatkowo bardzo łatwo się go instaluje w większości dystrybucji Linuksa. Dla dystrybucji Debian będzie to polecenie:

\begin{center}
\begin{verbatim}
	aptitude install make
\end{verbatim}
\end{center}

Po uruchomienia programu make, szuka on domyślnie, o ile nie podaliśmy w parametrach innej nazwy - pliku Makefile. Plik ten opisuje zależności miedzy plikami źródłowymi, i umożliwia skompilowanie tylko tych plików które uległy zmianie od ostatniego budowania. Zdecydowanie przyśpiesza to prace nad projektem w przypadku kiedy jest potrzebne testowania poprawek na bieżąco. 

Domyślne wywołanie make bez żadnych parametrów spowoduje zawsze uruchomienie domyślnego celu budowania: all. Możliwe jest definiowania dowolnej ilości celów budowania w jednym Makefile-u. Dzięki zastosowaniu takiego podejścia nie ma potrzeby edycji Makefilu kiedy chce skompilować aplikacje np. z innymi flagami kompilatora. Wystarczy wtedy dopisać odpowiedni cel budowania i go uruchomić. W projekcie Astro Rush został dodany również cel budowania służący do czyszczenia gry z wszystkich skompilowanych źródeł oraz z linkowanej aplikacji, co okazuje się przydatne kiedy występuje potrzeba przebudowania całego projektu. Tutaj należy wspomnieć o tym że Makefile sprawdza które pliki wymagają ponownej kompilacji, i kompiluje tylko te które uległy zmianie od ostatniego budowanie. Takie podejście znacząco skraca czas który programista musi poświecić na budowaniu aplikacji. Dodatkowo make wspiera kompilacje wielowątkowa, dzięki czemu na maszynach z procesorem wielordzeniowym można wykorzystać wszystkie rdzenie, skracając jeszcze bardziej czas budowy programu. 

Napisany na potrzeby gry plik „Makefile” pokazuje również że w pliku reguł programu make można definiować swoje zmienne. Taką zmienną może być na przykład ścieżka do kompilatora, która może się zmienić w zależności od platformy. Takimi zmiennymi mogą być również flagi linkera, kompilatora, oraz lista plików źródłowych z katalogu src. Tak napisany Makefile okazał się bardzo uniwersalny i dodanie kolejnych nowych plików źródłowych wymaga jedynie dopisanie ich na listę w zmiennej Makefile-a. 

Budowanie aplikacji poprzez make bądź też podobne narzędzie o nazwie CMake jest wyjątkowo popularne w systemie Linuks i projektach napisanych w języku C/C++. Make jest nawet wykorzystywany do budowania jądra Linuksa, które składa się z milionów linii kodu ( wersja 3.2 to w przybliżeniu 15 mln ) oraz setek plików, gdzie oszczędność czasu w przypadku rekompilacji jest bardzo znacząca kiedy kompilowane są tylko te moduły które zostały zmienione. Plik Makefile w grze Astro Rush wygląda następująco:
\begingroup
\fontsize{10pt}{12pt}\selectfont
\begin{verbatim}  
CXX = g++
CFLAGS = -Wall -pedantic -g -std=c++0x -I ./include -O0 -c

# flagi linkera
LIBS = -lGL -lGLU -lSDL -lSDL_mixer -lSDL_ttf -lSDL_image -lluabind -llua5.1

# lista plikow źródłowych do kompilacji
SOURCES = src/main.cpp src/App.cpp src/Property.cpp src/Resource.cpp 

# jak maja się nazywać skompilowanie pliki cpp
OBJECTS=$(SOURCES:.cpp=.o)

# nazwa pliku wynikowego
EXECUTABLE = AstroRush.bin

# domyslny cel dla wywolania make bez argumentu, czyli zbudowanie projektu
all: $(SOURCES) $(EXECUTABLE)

# linkowanie aplikacji
$(EXECUTABLE): $(OBJECTS)
	 @echo "\n ---- Linkowanie ---- "
	 $(CC) $(OBJECTS) -o $(EXECUTABLE) $(LIBS)

#kompilowanie plikow cpp
.cpp.o:
	 @$(CXX) $(CFLAGS) $< -o $@

# czyszczenie aplikacji przed zbudowaniem  
clean:
	rm -rf ./src/*.o
	rm ./AstroRush.bin 

\end{verbatim}  
\endgroup

%============================================================================================================================
% Git
%============================================================================================================================

\subsection{Git - system kontroli wersji}

Projekt Astro Rush nie wydaje się zbyt duży biorąc pod uwagę fakt że nie przekroczył 10 tysięcy linii kodu. Jednak zawsze warto mieć jakąś kopie na repozytorium oraz ewentualnie możliwość poprzez historie  zmian przywrócenie jakiś fragmentów kodu. Narzędziem które okazało się tutaj pomocne jest rozproszony system kontroli wersji – git. Darmowe oprogramowanie stworzone przez Linusa Torvaldsa do zarządzania kodem jądra Linuksa. Git w założeniu ma być pomocny w pracy zespołowej nad dużymi projektami, jednak w przypadku tak małego projektu jak Astro Rush również okazał się pomocny. Git jest również narzędziem wieloplatformowym, chodź pod systemem Windows jego klient jest on wolniejszy niż na Linuksie.

\begin{figure}[h]
    \centering
    \includegraphics[width=0.8\textwidth,natwidth=490,natheight=142]{./Pictures/git.png}
    \caption{Portal github.com. Podgląd pliku źródłowego. }
\end{figure}

Przy tworzeniu aplikacji został wykorzystany serwis hostujący gita: https://github.com. Portal posiada wiele funkcji wspomagających pracę zespołową nad projektem. Pozwala między innymi na edycję plików źródłowych bezpośrednio na serwerze, przeglądanie i porównywanie zmian w plikach. Dzięki czemu łatwo można sprawdzić co było w poprzedniej wersji pliku, oraz kto wysłał zmiany (ang.commit) na serwer. Github dostarcza statystyk na temat projektu np. wykres jak zmieniała się ilość kodu wraz z kolejnymi wysłanymi wersjami na serwer. Ponadto można tworzyć na serwerze dokumentacje do aplikacji (tzw. Wiki) którą następnie użytkownik może przeglądać przez przeglądarkę, bez potrzeby ściągania na dysk. Hosting dla aplikacji open source jest darmowy, jednak w takim przypadku repozytorium z kodem jest publiczne. Oznacza to że każdy może sobie ściągnąć projekt jednym poleceniem:

\begin{verbatim}
	git clone git://github.com/lolciuuu/astro.git
\end{verbatim}

Analogicznie z poziomu konsoli można później wysyłać zmiany na serwer (o ile dany użytkownik został dodany na listę osób które mogą zmieniać pliki na serwerze ), bądź też pobierać zmiany wysłanych przez innych użytkowników. Warto wspomnieć o tym że serwis github mimo tego że powstał dość nie dawno bo w 2008, ma już 2 miliony repozytoriów. 

Git w środowisku Linuks jest narzędziem konsolowym (W systemie Debian wystarczy zainstalować pakiet o takie samej nazwie), jednak w ramach ułatwienia podczas tworzenia aplikacji została wykorzystana wtyczka do Eclipse która pozwala w łatwy sposób synchronizować projekt który znajduje się na dysku lokalnie z tym co jest na repozytorium, oraz wysyłanie zmian na serwer. Dzięki wtyczce do tej podczas tworzenia projektu nie było potrzeby wydawać poleceń z poziomu konsoli, wszystko można wykonać z poziomu GUI.


\section{OpenGL jako silnik grafiki}
\lhead{Rozdział 1. \emph{OpenGL jako silnik grafiki}}

%============================================================================================================================
% Czym jest OpenGL
%============================================================================================================================
\subsection{Czym jest OpenGL}
Open Graphics Library (w skrócie OpenGL) jest niskopoziomową biblioteką  graficzną 3D. Kompatybilny jest on z większością liczących się systemów
operacyjnych, można z niego korzystać również na urządzeniach mobilnych np. z Androidem. OpenGL jest często wykorzystywany jako podstawowe API przy tworzeniu silników do gier 3D przykładem może
być tutaj choćby nawet silnik ID tech znany min. z serii gier Quake. Mimo że OpenGL jest przystosowany do pracy z grafiką trójwymiarową to doskonale
można go wykorzystać do grafiki 2D, tak jak to miało miejsce w grze Astro Rush. OpenGL posłużył do wyświetlania tekstur na ekranie, co odbywało się
zdecydowanie szybciej niż poprzez funkcje do rysowania z biblioteki SDL.  Różnica w szybkości renderowania wynosiła około 20 fps-ów (ang. frame per second ) na
laptopie hp550 z procesorem dual core 1.4 ghz. OpenGL dostarcza także takich zaawansowanych elementów jak obsługa cieni oraz oświetlenia, jednak z
racji wykorzystania grafiki 2D nie znalazło to zastosowania w projekcie.
	W bardzo łatwy sposób można połączyć SDL-a i OpenGL-a.
W SDL podstawowym elementem graficznym na którym odbywa się rysowanie jest powierzchnia (ang. surface). Podczas inicjowania biblioteki SDL tworzona
jest główna powierzchnia na której następnie będzie się odbywać rysowanie ( często też nazywane w grafice 2D „blitowaniem” ). Podczas inicjowania
głównej powierzchni ekranu żeby używać do renderowania OpenGL-a wystarczy poprzez funkcje SDL\_SetVideoMode podać flagę SDL\_OPENGL. Trzeba jeszcze
pamiętać żeby po każdym rysowaniu wywołać funkcję: SDL\_GL\_SwapBuffers() która wysyła bufor ramki do rysowania na ekranie.


%============================================================================================================================
% Wykorzystanie OpenGL do renderowania grafiki w grze
%============================================================================================================================
\subsection{Wykorzystanie OpenGL do renderowania grafiki w grze}
SDL posiada rozszerzenie SDL\_image które umożliwia obsługę różnych formatów grafiki (w grze zostały wykorzystane pliki graficzne o rozszerzeniach png oraz jpeg). Pozwala one w łatwy sposób wczytać plik z dysku poprzez funkcje SDL\_Surface *IMG\_Load(const char *file) , która zwraca surface z
wczytanym obrazkiem. SDL\_Surface jest strukturą w której znajduje się wskaźnik do pamięci gdzie znajduje się wczytana z dysku grafika. Ten adres (
„void* pixels” ) należy przekazać jedynie do OpenGL-a podczas tworzenia tekstury.
Można w ten sposób rysować nie tylko pliki graficzne wczytane z dysku ale również obiekty graficzne utworzone w aplikacji. Taka sytuacja ma miejsce w
przypadku renderowania napisów, gdzie jest tworzony surface z napisem, i następnie rysowany za pomocą OpenGL-a. W aplikacji proste prymitywy graficzne
jak prostokąty wypełnione kolorem są rysowane również za pomocą API OpenGL-a.


%============================================================================================================================
% Skrypty w języku Lua
%============================================================================================================================
\section{Skrypty w języku Lua}
\lhead{Rozdział 1. \emph{Skrypty w języku Lua}}
\hspace{1cm} Lua jest lekkim językiem skryptowym zaprojektowanym do rozszerzania możliwości innych aplikacji. Został on zaimplementowany w języku C zgodnie ze standardem ANSI, zapewnia mu to przenośność na wiele platform. Najważniejszym cechami tego języka jest to że jest dynamicznie typowany, oraz obiektowy. Jest on wykorzystywany zarówno do tworzenia rozszerzeń do różnych aplikacji, co często nazywane jest skryptowaniem (ang.scripting) oraz jako samodzielny język, w którym skrypty będą wykonywane poprzez maszynę wirtualną Lua. Samo skryptowanie wiąże się z ideą programowania sterowanego danymi (ang. data driven development).
Podejście te zakłada że wszelkie stałe kontrolujące zachowanie programu oraz wybrane elementy logiki powinny być zdefiniowane poza programem. W aplikacji skrypty Lua są wykorzystywane do przechowywanie wszystkich ustawień, oraz obliczania pewnych wartości 
już w czasie działania aplikacji.Przykładowo rozmiar gracza jest obliczany na poziomie skryptu przy wykorzystaniu ustawionych podczas działania gry zmiennych z rozmiarami ekranu. 

\begin{figure}[h]
    \centering
    \includegraphics[width=0.8\textwidth,natwidth=410,natheight=142]{./Pictures/lua_skrypty.png}
    \caption{Przepływ danych między aplikacją C++ a skryptem Lua}
\end{figure}

Powyższy rysunek pokazuje przykładowy przepływ danych między aplikacją C++ a skryptem Lua. Wykorzystane tu zostało wywołanie funkcji C++ z poziomu skryptu, aczkolwiek w drugą stronę ten mechanizm również działa. To znaczy z poziomu C++ można wykonać funkcje Lua. Lua została wykorzystana do internacjonalizacji aplikacji. Podobnie jak to jest wykorzystywane w aplikacjach webowych w języku Java, gdzie wszystkie komunikaty są przechowywane w plikach properties. Podczas uruchamiania aplikacji zostaje odczytany odpowiedni plik dla danego języka. Cały proces ilustruje rysunek 5. W rezultacie takiej budowy aplikacja może działać z dowolnym językiem. Podczas tworzenia zostały napisane komunikaty zarówno polskie jak i angielskie.


%============================================================================================================================
% Wprowadzenie do biblioteki SDL
%============================================================================================================================
\section{Wprowadzenie do biblioteki SDL}
\lhead{Rozdział 1. \emph{Wprowadzenie do biblioteki SDL}}
Rozdział ten będzie zawierał wprowadzenie do tworzenia aplikacji z wykorzystaniem biblioteki SDL, nie będzie tu poruszony temat instalacji tej biblioteki. Zagadnienie te będzie znajdować się w rozdziale dotyczącym kompilacji projektu w systemie Linuks. 

Simple Direct Media Layer jest biblioteką ułatwiającą tworzenie gier komputerowych, oraz różnych aplikacji multimedialnych. Umożliwia ona stworzenie okna, oraz zarządzanie nim. Dodatkowo zapewnia obsługę zdarzeń związanych z klawiaturą, myszą oraz joystickiem. Możliwa jest nawet obsługa CD-ROM-u za pomocą tego API. Największą zaletą obok dużej funkcjonalności jest prostota aplikacji pisanej z wykorzystaniem tej biblioteki. Utworzenie kontekstu graficznego wymaga jedynie kilku linijek, a najprostszy program typu "Hello Word" może wyglądać następująco:

\begingroup
\fontsize{10pt}{12pt}\selectfont
\begin{verbatim} 
#include <SDL/SDL.h>

int main(void)
{
    SDL_Init( SDL_INIT_VIDEO );
    SDL_Surface * screen = SDL_SetVideoMode( 800, 600, 32, SDL_SWSURFACE );
    SDL_Flip( screen );
    SDL_Quit();
    return 0;
}
\end{verbatim}
\endgroup
Tak napisany kod można skompilować przy pomocy kompilatora g++ z poziomu linuksowej konsoli za pomocą polecenia:
\begin{verbatim} 
	g++ ./sdl_simple.cpp -o sdl_simple -lSDL
\end{verbatim}

Aplikacja po uruchomieniu utworzy okno które zostanie natychmiast zamknięte. Tak działająca aplikacja jest mało przydatna, dlatego też konieczna jest pętla w której będzie odbywać się praca całego programu. Bardzo ważnym elementem takiej pętli jest obsługa zdarzeń, którą można zrealizować poprzez użycie funkcji SDL\_PollEvent, przyjmującej jako parametr adres do struktury SDL\_Event, którą to wypełnia informacjami o zdarzeniach które miały miejsce. Następnie za pomocą warunków logicznych należy sprawdzić jakie zdarzenia miały miejsce. np. warunek if(pEvent.type == SDL\_KEYDOWN \&\& pEvent.key.keysym.sym == SDLK\_ESCAPE)  
udzieli informacji czy miało miejsce naciśniecie klawisza escape. Analogicznie ze struktury SDL\_Event możemy wyciągnąć informacje o dowolnym klawiszu, naciśnięciu klawisza myszy, itp. 

SDL dostarcza również obsługę wielowątkowości oraz timerów. Dzięki czemu nie potrzebna była w projekcie żadna dodatkowa biblioteka która umożliwiała by utworzenie timera. Zadaniem timera jest uruchamianie pewnego zadania co stały określony czas, w przypadku biblioteki SDL jest wywoływana funkcja do której wskaźnik przekazujemy do funkcji  SDL\_AddTimer, drugim ważnymi parametrem tej funkcji jest czas milisekundach co ile mas się wykonać ta funkcja.Funkcja która rejestrujemy do wykonywania przez timer powinna (@TODO typ) oraz funkcja SDL\_AddTimer zwróci nad dane utworzonego timera w postaci struktury SDL\_TimerID. W taki właśnie sposób zostały wykorzystane timery w Astro Rush do np wyłączania nieśmiertelności bohatera który zebrał i wykorzystał bonus. Żeby cały mechanizm timerów działał potrzebne jest zainicjowanie go przy tworzeniu okna poprzez użycie flagi SDL\_INIT\_TIMER.
\begin{verbatim} 
SDL_INIT_TIMER SDL_TimerID
onWithEnemy = false;
timer = SDL_AddTimer( DISABLE_COLLISION_WITH_ENEMY_TIME,
 enableEnemyDetect_callbackTimer, this );
static Uint32 enableEnemyDetect_callbackTimer(Uint32 interval, void *param);
\end{verbatim}

%============================================================================================================================
% Obsługa czcionek
%============================================================================================================================
\section{Obsługa czcionek w SDL}
\lhead{Rozdział 1. \emph{Obsługa czcionek w SDL}}
W projekcie została wykorzystana biblioteka SDL\_ttf pozwalająca używać w aplikacji czcionek w formacie True Type. Format ten stworzony przez firmę Apple przechowuje kształty poszczególnych liter jako krzywe Beziera, i jest on obsługiwany przez większość platform. Na Linuksie jest on bardzo powszechnym formatem do obsługi czcionek, dodatkowym plusem jest ogromna ilość czcionek na darmowych licencjach. W grze została wykorzystana czcionka "Ubuntu" udostępniona za darmo, i będąca domyślną czcionką w dystrybucji Linuksa o tej samem nazwie. Plik z taką czcionką (standardowo o rozszerzeniu *.ttf) jest wczytywany podczas uruchamiania aplikacji, następnie poprzez wywołania funkcji z biblioteki SDL\_ttf np. TTF\_RenderUTF\_Blended zostaje utworzona powierzchnia na której narysowany jest napis o podanej treści, kolorze oraz rozmiarze. 

Niestety powierzchnia taka jest zwracana jako wskaźnik na strukturę SDL\_Surface, przez co konieczna jest konwersja na format obsługiwany przez OpenGL-a. Niedogodność taka nie występowałaby gdyby do renderowania było wykorzystywane API biblioteki SDL. Identyczny problem występuje również przy renderowaniu innych elementów graficznych które są ładowane z dysku i zwracane jako SDL\_Surface* (Wczytywanie takie realizowane jest poprzez kolejną bibliotekę będącą uzupełnieniem SDL-a: SDL\_image. Służy ona do wczytywania plików graficznych w takich formatach jak np. JPEG, PNG, TIFF. W aplikacji wykorzystana jest tylko jedna funkcja 
z tej biblioteki stąd też nie będzie ona szerzej omawiana).

Sama konwersja SDL\_Surface* na GLuint to wygenerowanie tekstury w standardowy dla OpenGL-a sposób, wykorzystując przy tym pole pixels ze struktury SDL\_Surface, które jest adresem pod którym przechowywane są poszczególne piksele obrazka. W uproszczeniu funkcja realizująca taką konwersje w grze wygląda następująco:

\begingroup
\fontsize{10pt}{12pt}\selectfont
\begin{verbatim}  
void RendererGL::create_gl(SDL_Surface * surf, GLuint * tex )
{
 
   /** ...tutaj określenie ilości kolorów i formatu */
  
    glGenTextures( 1, tex );
    glBindTexture( GL_TEXTURE_2D, *tex );

    /** ...tutaj ustawienia parametrów tekstury */

    glTexImage2D( GL_TEXTURE_2D, 0, colors_amount,
    			  	surf->w, surf->h, 0, format, 
    			 	 GL_UNSIGNED_BYTE, surf->pixels );
}
\end{verbatim}
\endgroup

Warto wspomnieć że biblioteka SDL\_ttf pozwala renderować napisy z polskimi znakami, o ile takie występują w wczytanej czcionce. Ponadto SDL\_ttf dostępny jest podobnie jak SDL na darmowej licencji zlib, i jest wieloplatformowy jak wszystkie wtyczki do SDL-a. W niektórych dystrybucjach zainstalowanie tej biblioteki, oraz innych wspomnianych bibliotek rozszerzających SDL-a sprowadza się do wykonania jednego polecenia- zainstalowania pakietu z repozytorium. Dla dystrybucji Debian oraz jego pochodnych będzie to polecenie:
\begin{verbatim}
apt-get install libsdl-ttf2.0-dev 
\end{verbatim}

W przypadku innych dystrubucji kod źródłowy biblioteki można pobrać ze strony
\href{http://www.libsdl.org/projects/SDL_ttf/}{www.libsdl.org}.
Na stronie tej dostępne są również prekompilowane pakiety dla dystrybucji opartych na pakietach rpm.
\chapter{Budowa aplikacji} 
%============================================================================================================================
%							 						Glowna petla
%============================================================================================================================

\section{Główna pętla}
\lhead{Rozdział 2. \emph{Główna pętla}}

W grach komputerowych często wykorzystywane jest tzw. programowanie sterowane zdarzeniami. Polega ono na umieszczeniu w aplikacji głównej pętli, w której to cyklicznie będzie się odbywać obsługa zdarzeń (np. naciśniecie klawisza ), aktualizacja gry oraz rysowanie. Sama kolejność tych elementów nie odgrywa większej roli, warto natomiast zwrócić uwagę na to że po zatrzymaniu pętli następuje przygotowanie aplikacji do wyłączenia. W przypadku Atsro Rush po wyjściu z głównej pętli zatrzymywany jest kontekst graficzny SDL-a, zwalniane są wszystkie zajęte zasoby, i następuje wyłączenie gry. Pętla taka najczęściej implementowana jest jak while, którego zakończeniem steruje flaga wyjścia. Przy każdym obiegu pętli wartość tej flagi wyciągana jest z klasy Game, która to decyduje kiedy należy zakończyć działanie aplikacji. 

\begin{figure}[h]
    \centering
    \includegraphics[width=0.8\textwidth,natwidth=410,natheight=142]{./Pictures/main_loop.png}
    \caption{Schemat działania głównej pętli}
\end{figure}

Pętla stanowi najważniejszy element większości gier, to od niej zależy czy gra będzie działać tak samo na urządzeniach różniących się wydajnością. W projekcie pętla jest dość prosta i oprócz typowych elementów typu aktualizacja, rysowanie, obsługa zdarzeń, uwzględnia jedynie sytuacje w której całość obliczeń i rysowań odbywa się zbyt szybko i należy wykonać opóźnienie. Taki problem może się pojawić na szybszych urządzeniach na których gra działała by zbyt szybko. W przypadku bardziej złożonej aplikacji można także uwzględnić sytuacje odwrotną, kiedy to ostatnia aktualizacja stanu gry odbywała się zbyt wolno i w następnym obiegu pętli należy wykonać aktualizacje kilkukrotnie żeby zapobiec braku płynności w renderowanym obrazie. Taka sytuacja jednak w grze Astro Rush nie powinna mieć miejsca z racji tego iż jest jest to gra z grafiką dwuwymiarowa i występują w niej proste obliczenia matematyczne.


%============================================================================================================================
%							 						Mapa kafelkowa
%============================================================================================================================
\section{Mapa w grze}

\lhead{Rozdział 2. \emph{Mapa kafelkowa}}
\subsection{Mapa kafelkowa}
Mapa kafelkowa (ang. tiled map) jest jedną z podstawowych technik przy tworzeniu gier z grafiką dwuwymiarową. Technika ta polega na podziale świata dostępnego w grze na fragmenty (tzw. kafelki ) o tych samych rozmiarach. Najczęściej są to kwadraty, którym przypisujemy odpowiednie identyfikatory grafik. Tak utworzona mapa przechowywana jest w postaci dwuwymiarowej macierzy w osobnym pliku na dysku, i jest wczytywana podczas startu aplikacji w osobnym wątku. Macierz składająca się wyłącznie z cyfr (typu short żeby dodatkowo oszczędzić pamięć) zajmuje o wiele mniej pamięci w przeciwieństwie do rozwiązania w którym  z dysku wczytywana jest cała mapa w postaci jednej grafiki. Na podstawie tej macierzy rysowana jest mapa widoczna na ekranie. 

Z racji tego że gracz ciągle wędruje prze mapę, ta cały czas jest przesuwana, a dokładniej  to inkrementowany jest indeks kolumny w macierzy kafelków od której zaczynamy rysowanie. Kolumna o takim indeksie rysowana jest na ekranie jako pierwsza z lewej strony, następnie rysowane są obok (po prawej stronie) kolejne kolumny aż do momentu w którym mapa pokrywa cały ekran.
Kolumna od której zaczyna się rysować od lewego brzegu ekranu przesunięta jest w lewo o pewien offset, który zwiększnay jest podczas biegu gracza do przodu. Offset ten sprawia że współrzędna na osi X skrajnej kolumny zostaje przesunięta w lewo po za ekran, tak że widoczny jest tylko fragment kolumny na ekranie. Przejście do następnej kolumny następuje w momencie kiedy skrajna kolumna znajduje się całkiem po za ekranem. Dzięki zastosowaniu takiego algorytmu nastepuje płynne przesuwanie mapy, bez widocznych przeskoków pomiędzy kolejnymi kolumnami.
 

%============================================================================================================================
%							 						Edytor leveli
%============================================================================================================================
\subsection{Edytor mapy}
\lhead{Rozdział 2. \emph{Edytor mapy}}
Opisana w poprzednim podrozdziale macierz kafelków w grze Astro Rush ma wymiary: 30000 x 15. Stąd też pojawił się problem edycji tak dużej ilości danych. Zmiana poszczególnych wpisów ręcznie nie wchodziła w grę, dlatego też powstała dodatkowa aplikacja do edycji mapy. W wizualny sposób, z wykorzystaniem jedynie myszy można w niej stworzyć w kilkanaście minut całą mapę, rozmieszczając na niej dostępne rodzaje kafelków. Edytor umożliwia także wczytanie stworzonej wcześniej mapy i jej edycje. Aplikacja została napisana z wykorzystaniem biblioteki Qt udostępnionej na licencji LGPL. Biblioteka ta jest zestawem przenośnych narzędzi do tworzenia między innymi interfejsu użytkownika, obsługi sieci, grafiki trójwymiarowej (OpenGL), plików i wielu innych. 


\begin{figure}[h]
    \centering
    \includegraphics[width=0.8\textwidth,natwidth=800,natheight=160]{./Pictures/designer.png}
    \caption{Edytor mapy podczas pracy}
\end{figure}

Edytor mapy wyświetla całą planszę w postaci siatki na której naniesione są kafelki. Poprzez kliknięci w daną komórkę możemy zmienić rodzaj kafelka który w danym miejscu ma się wyświetlić, bądź też wyczyścić daną komórkę. Do pliku zapisywane są tylko numery odpowiadającym poszczególnym kafelkom, na bazie których gra rozpoznaje jaką grafikę w danym miejscu wstawić. Numeracja kafelków rozpoczyna się od 0, natomiast wartość -1 oznacza że w danym miejscu nie ma kafelka i taki fragment nie jest rysowany. Edytor jest aplikacją bardzo prostą, wszelkie jego modyfikacje np. dodanie nowego rodzaju kafelka wymaga ręcznych zmian w kodzie programu, jednak na potrzeby pracy takie rozwiązanie okazało się wystarczające.

%============================================================================================================================
%							 				Uruchamianie aplikacji
%============================================================================================================================
\section{Uruchamianie aplikacji}
\lhead{Rozdział 2. \emph{Uruchamianie aplikacji}}
Aplikacje takie jak gry wymagają do pracy zewnętrznych zasobów które są przechowywane na dysku. Mogą to być grafiki, dźwięki, czcionki. Wraz ze wzrostem ilości materiałów jakie muszą być załadowane do aplikacji przy jej starcie rośnie też czas oczekiwania gracza na to aż aplikacja będzie gotowa do pracy. Dlatego też wczytywanie zasobów, oraz inne czynności które trwają długo są wykonywane podczas uruchamiania, a użytkownikowi prezentowany jest w tym czasie ekran powitalny (ang. Splash screnn). W grze Astro Rush taki ekran również jest prezentowany. Pokazuje się na nim również pasek postępu obrazujący ile czasu pozostało jeszcze do uruchomienia właściwej części aplikacji. Żeby samo rysowanie ekranu powitalnego odbywało się płynnie wczytywanie danych z dysku odbywa się w osobnym wątku.

\begin{figure}[h]
    \centering
    \includegraphics[width=0.8\textwidth,natwidth=800,natheight=152]{./Pictures/splash.png}
    \caption{Schemat uruchamiania gry}
\end{figure}

Wielowątkowość w projekcie jest możliwa dzięki wykorzystaniu specjalnej funkcji z biblioteki SDL. Funkcja SDL\_CreateThread przyjmująca jako argument wskaźnik do funkcji tworzy nowy wątek w którym zostaje uruchomiona ta funkcja. W funkcji takiej zostało umieszczone wczytywanie wszystkich zasobów z dysku. W momencie kiedy funkcja wczyta wszystkie dane, wtedy ustawi flagę oznaczającą zakończenie ładowania na wartość True. Zmiana wartości tej flagi spowoduje zakończenie wyświetlania ekranu powitalnego i uruchomi główną pętle gry. Cała funkcjonalność związana z ekranem powitalnym została umieszczona w osobnej klasie Splash która jest dziedziczona prze główną klasę aplikacji - App. Dzięki takiej implementacji klasa App zajmuję się jedynie inicjowaniem bibliotek, oraz obsługą głównej pętli. 

Należy wspomnieć ile miejsca zajmują omawiane zasoby na dysku. W przypadku  dźwięków jest to około 10 mb, grafika to 3 mb, a sam plik z mapą to 1.3 mb. Na wczytanie takich ilości danych w zależności od sprzętu może być potrzebne nawet do kilku sekund.



%============================================================================================================================
%							 		Przyjęte konwencje, logowanie i obsługa błedów
%============================================================================================================================
\section{Przyjęte konwencje, logowanie i obsługa błedów}
\lhead{Rozdział 2. \emph{Przyjęte konwencje, logowanie i obsługa błedów}}
Lorem ipsum dolor sit amet, consectetur adipiscing elit. Duis eu massa ante. Maecenas pretium metus a libero commodo convallis. Mauris a dignissim lacus. Cum sociis natoque penatibus et magnis dis parturient montes, nascetur ridiculus mus. Duis eleifend magna ut magna commodo dapibus. In adipiscing enim eget sapien elementum et adipiscing ligula sagittis. Curabitur ullamcorper cursus vulputate. Donec dignissim, tortor eget adipiscing rhoncus, risus mauris varius nisi, ac vehicula elit orci sit amet nunc. Fusce massa nisi, imperdiet vitae volutpat non, euismod ullamcorper lectus. Mauris iaculis sagittis tortor, quis convallis elit luctus eu. Sed sodales viverra velit, quis porttitor ipsum vulputate nec.



%============================================================================================================================
%							 					Warstwy aplikacji
%============================================================================================================================
\section{Warstwy aplikacji}
\lhead{Rozdział 2. \emph{Warstwy aplikacji}}

\begin{figure}[h]
    \centering
    \includegraphics[width=430px,height=560px]{./Pictures/warstwy.png}
    \caption{Schemat przepływu danych w aplikacji}
\end{figure}

Lorem ipsum dolor sit amet, consectetur adipiscing elit. Duis eu massa ante. Maecenas pretium metus a libero commodo convallis. Mauris a dignissim lacus. Cum sociis natoque penatibus et magnis dis parturient montes, nascetur ridiculus mus. Duis eleifend magna ut magna commodo dapibus. In adipiscing enim eget sapien elementum et adipiscing ligula sagittis. Curabitur ullamcorper cursus vulputate. Donec dignissim, tortor eget adipiscing rhoncus, risus mauris varius nisi, ac vehicula elit orci sit amet nunc. Fusce massa nisi, imperdiet vitae volutpat non, euismod ullamcorper lectus. Mauris iaculis sagittis tortor, quis convallis elit luctus eu. Sed sodales viverra velit, quis porttitor ipsum vulputate nec.



\chapter{Dokumentacja użytkownika} 
%============================================================================================================================
%							 					Sterowanie w grze
%============================================================================================================================
\section{Sterowanie w grze}
\lhead{Rozdział 1. \emph{Fabuła}}
Lorem ipsum dolor sit amet, consectetur adipiscing elit. Duis eu massa ante. Maecenas pretium metus a libero commodo convallis. Mauris a dignissim lacus. Cum sociis natoque penatibus et magnis dis parturient montes, nascetur ridiculus mus. Duis eleifend magna ut magna commodo dapibus. In adipiscing enim eget sapien elementum et adipiscing ligula sagittis. Curabitur ullamcorper cursus vulputate. Donec dignissim, tortor eget adipiscing rhoncus, risus mauris varius nisi, ac vehicula elit orci sit amet nunc. Fusce massa nisi, imperdiet vitae volutpat non, euismod ullamcorper lectus. Mauris iaculis sagittis tortor, quis convallis elit luctus eu. Sed sodales viverra velit, quis porttitor ipsum vulputate nec.


%============================================================================================================================
%							 						Menu
%============================================================================================================================
\section{Menu}
\lhead{Rozdział 1. \emph{Fabuła}}
Lorem ipsum dolor sit amet, consectetur adipiscing elit. Duis eu massa ante. Maecenas pretium metus a libero commodo convallis. Mauris a dignissim lacus. Cum sociis natoque penatibus et magnis dis parturient montes, nascetur ridiculus mus. Duis eleifend magna ut magna commodo dapibus. In adipiscing enim eget sapien elementum et adipiscing ligula sagittis. Curabitur ullamcorper cursus vulputate. Donec dignissim, tortor eget adipiscing rhoncus, risus mauris varius nisi, ac vehicula elit orci sit amet nunc. Fusce massa nisi, imperdiet vitae volutpat non, euismod ullamcorper lectus. Mauris iaculis sagittis tortor, quis convallis elit luctus eu. Sed sodales viverra velit, quis porttitor ipsum vulputate nec.


%============================================================================================================================
%							 						Instalacja
%============================================================================================================================
\section{Instalacja}
\lhead{Rozdział 1. \emph{Fabuła}}
Lorem ipsum dolor sit amet, consectetur adipiscing elit. Duis eu massa ante. Maecenas pretium metus a libero commodo convallis. Mauris a dignissim lacus. Cum sociis natoque penatibus et magnis dis parturient montes, nascetur ridiculus mus. Duis eleifend magna ut magna commodo dapibus. In adipiscing enim eget sapien elementum et adipiscing ligula sagittis. Curabitur ullamcorper cursus vulputate. Donec dignissim, tortor eget adipiscing rhoncus, risus mauris varius nisi, ac vehicula elit orci sit amet nunc. Fusce massa nisi, imperdiet vitae volutpat non, euismod ullamcorper lectus. Mauris iaculis sagittis tortor, quis convallis elit luctus eu. Sed sodales viverra velit, quis porttitor ipsum vulputate nec.


\begin{lstlisting}
if( int x =0 ) {
	cout<<dupa;
}
\end{lstlisting}

% wstep
\setcounter{secnumdepth}{-1}
\renewcommand{\chaptername}{}
\lhead{\emph{Zakończenie}}
\chapter{Zakończenie} 
%============================================================================================================================
%Zakończenie
%============================================================================================================================
\hspace{1cm} W pracy została przedstawiona budowa platformowej gry zręcznościowej mogącej pracować w systemie Linuks, gdzie brakuje gier oraz w najpopularniejszym obecnie systemie dla komputerów osobistych- Windows. 
Gra została stworzona w oparciu o własny silnik, pokazujący możliwości biblioteki SDL. Zaczynając od możliwości stworzenia i zarządzania oknem, poprzez integracje z OpenGL aż do obsługi dźwięku oraz czcionek. Omówiona biblioteka nie ustępuje możliwością bibliotece DirectX, której ogromnym ograniczeniem jest brak wsparcia dla systemu Linuks. Ponadto aplikacja pokazuje przykładowa budowę gry w której do renderowania grafiki wykorzystuje się sprite-y oraz mape kafelkową. Techniki te są jednymi z najbardziej podstawowych zagadnien w grach opartych o grafike dwuwymiarową.

W pracy został poruszony temat rozszerzenia możliwości programu w oparciu o tzw. skryptowanie. Zaprezentowane skrypty w języku Lua stanowią obecnie wtyczki rozszerzające funkcjonalonść nie tylko w grach, lecz wszędzie tam gdzie przeniesienie pewnych informacji poza skompilowany program pozwala oszczędzić czas potrzebny na jego napisanie.  

Stworzona aplikacja nie musi być tylko programem napisanym do pracy dyplomowej, może zostać ona upowszechniona jako projekt Open Source. Nie wykluczone jest, że wtedy będzie dalej rozwijana przez osoby chętne do wolontariatu. Udostępnienie kodu jako darmowego pozwala również na stworzenie pakietu prekompilowanego dla systemu Debian ( Plik z rozszerzeniem deb) i zgłoszenie go do repozytorium dystrybucji Debian, bądź Ubuntu. 

Inną możliwością co do przyszłością programu jest sprzedawanie gry jako tzw. Indie Game czyli gry niezależnej, stworzonej bez wsparcia finansowego przez jedną bądź kilka osób. Taką aplikacje można sprzedawać w formie elektronicznej za niewielkie pieniądze. Aplikacja może zostać uruchomiona na tablecie z systemem BlackBerry i umieszczona w markecie z aplikacjami dla tego systemu. Mowa jest tutaj szczególnie o tabletach ponieważ aplikacja podlega ograniczeniom odnośnie rozdzielczości ekranu na którym może zostać uruchomiona, co zostało opisane bardziej szczegółowo w pracy.  

Efektem ubocznym stworzenia gry Astro Rush było stworzenie silnika gry 2D. Silnik ten dzięki zastosowaniu zewnętrznych skryptów podatny jest na modyfikacje i zmianę. Prosta budowa pozwala na stworzenie na bazie bieżącej aplikacji kolejnej gry. Użycie biblioteki OpenGL do renderowania grafiki daje możliwość stworzenia dużo bardziej zaawansowanej grafiki, wykorzystującej cieniowanie i oświetlenie. 



\lhead{\emph{Spis ilustracji}} % Set the left side page header to "List of Figures"
\listoffigures % Write out the List of Figures
\backmatter
%----------------------------------------------------------------------------------------
%									Bibliografia
%----------------------------------------------------------------------------------------
 \begin{thebibliography}{1}

\bibitem{sop}Janusz Ganczarski. 
\emph{OpenGL w praktyce}. Wydawnictwo BTC, 2008.

\bibitem{sop}Ernest Pazera. 
\emph{Focus o SDL}. Premier Press, 2003

\end{thebibliography}
\end{document}


\end{document} 
